\documentclass{article}

\usepackage{geometry}
\geometry{a4paper}
\setlength{\parindent}{10mm}
\setlength{\parskip}{0.9em}
\def\baselinestretch{1.5}

\usepackage[spanish,mexico]{babel}
\renewcommand {\spanishtablename}{Cuadro}
\usepackage[spanish,onelanguage,ruled]{algorithm2e}
\usepackage[utf8]{inputenc}
\usepackage{graphicx}
\usepackage{caption} 
\usepackage{amsmath}
\usepackage{amsthm, amsfonts}
\usepackage{enumerate} 
\usepackage{fancyhdr}
\usepackage{anysize} 
\usepackage[usenames]{color}
\usepackage{booktabs}
\usepackage{etoolbox}
\usepackage{fancyvrb}
\usepackage{color,soul}
\usepackage[dvipsnames]{xcolor}
\usepackage{graphicx}
 \usepackage{caption}
\usepackage{subcaption}
\usepackage{listings}
\usepackage{amssymb}
\usepackage{multirow}
\usepackage{dirtytalk}



\usepackage{verbatim}
% redefine \VerbatimInput
\RecustomVerbatimCommand{\VerbatimInput}{VerbatimInput}%
 {fontsize=\footnotesize,
  %
  frame=lines,  % top and bottom rule only
  framesep=1em, % separation between frame and text
  rulecolor=\color{Gray},
  %
  label=\fbox{\color{Black}test.txt},
  labelposition=topline,
  %
  %commandchars=\ \mid \(\), % escape character and argument delimiters for
                  % commands within the verbatim
  %commentchar=*        % comment character
 }


\pagestyle{fancy}

\chead{}
\lhead{} 
\rhead{}
\lfoot{\it }
\cfoot{}
\rfoot{\thepage}

\title{
\centering
Modelos Probabilistas Aplicados \\
Johanna Bolaños Zúñiga \\
Matricula: 1883900\\
Tarea 15
}

\date{}

\begin{document}
\maketitle

\section{Propuestas para el proyecto final de la clase}

Las siguientes propuestas son planteadas como ideas para realizar el producto integrador de la materia Modelos Probabilistas Aplicados, las cuales podrían ser modificadas para llevar a cabo un trabajo final que cumpla con los requisitos:

\begin{enumerate}
    \item En el tema de tesis, se cuenta con un modelo matemático y una metaheurística para encontrar la solución del problema planteado. Por medio de la prueba de hipótesis de medias de diferencia se pretende comprobar que existe un ahorro entre la metaheurística y la solución ofrecida por el modelo matemático en un $95\%$ y determinar que tanto mejora la solución considerando intervalos de confianza del $90\%$ y $95\%$. Estas pruebas se aplicarán tanto por tamaño de instancias (pequeñas, medianas tipo $1$, medianas tipo $2$ y grandes) como para el total de instancias. Además, se aplicaría la prueba de hipótesis para proporciones para determinar la proporción de mejores soluciones encontradas mediante el uso de la metaheurística.
    \item Explicar el valor de PIB en Colombia en función de variables como la tasa de interés, inversiones, número de empleados y tasas de cambio, utilizando un modelo de regresión lineal múltiple con la técnica de Mínimos Cuadrados Ordinarios (MCO).
    \item Mediante información de la Encuesta Nacional de Niños, Niñas y Mujeres (ENIM) 2015 en México, se construye la variable binaria que tomará el valor de 1 si el niño (entre 0 y 5 años) está desnutrido y cero, en caso contrario. La desnutrición es medida a través del indicador talla para la edad. Se utilizarán variables explicativas como la edad de la madre, número de hermanos, región de nacimiento y lactancia por parte de la madre, ingresos de la familia y, se utilizará un modelo de regresión Logit para tratar de explicar la probabilidad de si un niño es desnutrido o no.
    \item Realizar un diseño de experimentos para evaluar entre un grupo de personas (por medio de una encuesta) la probabilidad estimada que tienen de ingresar a la Universidad considerando factores como la edad, genero, raza, ingresos mensuales, experiencia laboral, años de educación y si están trabajando.
\end{enumerate}


%\bibliography{refProbabilidad}
%\bibliographystyle{plain}

\end{document}

