\documentclass{article}

\usepackage{geometry}
\geometry{a4paper}
\setlength{\parindent}{10mm}
\setlength{\parskip}{0.9em}
\def\baselinestretch{1.5}

\usepackage[spanish,mexico]{babel}
\renewcommand {\spanishtablename}{Cuadro}
\usepackage[spanish,onelanguage,ruled]{algorithm2e}
\usepackage[utf8]{inputenc}
\usepackage{graphicx}
\usepackage{caption} 
\usepackage{amsmath}
\usepackage{amsthm, amsfonts}
\usepackage{enumerate} 
\usepackage{fancyhdr}
\usepackage{anysize} 
\usepackage[usenames]{color}
\usepackage{booktabs}
\usepackage{etoolbox}
\usepackage{fancyvrb}
\usepackage{color,soul}
\usepackage[dvipsnames]{xcolor}
\usepackage{graphicx}
 \usepackage{caption}
\usepackage{subcaption}
\usepackage{listings}
\usepackage{amssymb}
\usepackage{multirow}



\usepackage{verbatim}
% redefine \VerbatimInput
\RecustomVerbatimCommand{\VerbatimInput}{VerbatimInput}%
 {fontsize=\footnotesize,
  %
  frame=lines,  % top and bottom rule only
  framesep=1em, % separation between frame and text
  rulecolor=\color{Gray},
  %
  label=\fbox{\color{Black}test.txt},
  labelposition=topline,
  %
  %commandchars=\ \mid \(\), % escape character and argument delimiters for
                  % commands within the verbatim
  %commentchar=*        % comment character
 }


\pagestyle{fancy}

\chead{}
\lhead{} 
\rhead{}
\lfoot{\it }
\cfoot{}
\rfoot{\thepage}

\title{
\centering
Modelos Probabilistas Aplicados \\
Johanna Bolaños Zúñiga \\
Matricula: 1883900\\
Tarea 12
}

\date{}

\begin{document}
\maketitle

\section{Poblemas a resolver}

En el presente trabajo se realizaron las soluciones de diversos problemas del libro de Grinstead \cite{librop} sobre las funciones generadoras de momentos de una variable aleatoria (v.a), así como el valor esperado y varianza. Para el cálculo de estos valores se utilizaron las ecuaciones \ref{continua}, \ref{Vcontinua}, \ref{varianza}, donde $X$ es una v.a con distribución de probabilidad $f{(x)}$:

\noindent Función generadora de momentos sí $X$ es continua,
\begin{equation}
g{(t)} = E{(e^{xt})} = \int_{-\infty}^{\infty}e^{xt}f(x)\,dx.
\label{continua}
\end{equation}
\noindent Valor esperado si $X$ es continua,
\begin{equation}
\mu = E{(X)} = \int_{-\infty}^{\infty}xf(x)dx.
\label{Vcontinua}
\end{equation}
\noindent Varianza de $X$,
\begin{equation}
\sigma^2 = V(X)= E{(X^2)} - \mu^2
\label{varianza}
\end{equation}

\subsection{Problema 1, página 392}
\noindent \textit{Let $Z_{1}$, $Z_{2}$, $\dots$, $Z_{N}$ describe a branching process in which each parent has $j$ offspring with probability $p_{j}$. Find the probability $d$ that the process eventually dies out if:}

\begin{enumerate}[a)]
    \item $p_{0}$ = $1/2$, $p_{1}$ = $1/4$, $p_{2}$ = $1/4$
    \item $p_{0}$ = $1/3$, $p_{1}$ = $1/3$, $p_{2}$ = $1/3$
    \item $p_{0}$ = $1/3$, $p_{1}$ = $0$, $p_{2}$ = $2/3$
    \item $p_{j}$ = $1/2^{j+1}$
    \item $p_{j}$ = $(1/3)(2/3)^j$
    \item $p_{j}$ = $\frac{e^{-2}2^j}{j!}$
\end{enumerate}  

De acuerdo con el teorema 10.2, se tiene que sí $m \leq 1$, entonces $d=1$ y el proceso acaba con probabilidad $1$; sí $m > 1$, entonces $d < 1$ y el proceso acaba con probabilidad $d$. Para el cálculo del valor de $m$ se utiliza las siguientes expresiones:
\begin{align}
    \nonumber
    m & = p_{1}+2p_{2} = 1-p_{0}-p_{2}+2p_{2} = 1-p_{0}+p_{2} \\ \nonumber
    h(z) & = p_{0} + p_{1}z + p_{2}z^2 + \dots \\ \nonumber
    m & = h'(1). \nonumber
\end{align}

\noindent \textbf{Inciso a)}
\begin{align}
    \nonumber
        m & = \frac{1}{4} + 2\left(\frac{1}{4}\right) \\ \nonumber
        m & = \frac{3}{4}. \nonumber
\end{align}
Como $m \leq 1$ y $p_{0} > p_{2}$, entonces $d=1$ y el proceso acaba con una probabilidad de $1$.

\noindent \textbf{Inciso b)}
\begin{align}
    \nonumber
        m & = \frac{1}{3} + 2\left(\frac{1}{3}\right) \\ \nonumber
        m & = 1. \nonumber
\end{align}
Como $m \leq 1$ y $p_{0} = p_{2}$, entonces $d=1$ y el proceso acaba con una probabilidad de $1$.

\noindent \textbf{Inciso c)}
\begin{align}
    \nonumber
        m & = 0 + 2\left(\frac{2}{3}\right) \\ \nonumber
        m & = \frac{4}{3}. \nonumber
\end{align}
Como $m > 1$ y $p_{0} < p_{2}$, entonces $d<1$ y el proceso acaba con una probabilidad de $d$. Para el cálculo del valor de $d$, se utliza la ecuación \ref{valord}:
\begin{align} \label{valord}
        d & = \frac{p_{0}}{p_{2}}\\ \nonumber
        d & = \frac{\frac{1}{3}}{\frac{2}{3}}\\ \nonumber
        d & = \frac{1}{2} \\ \nonumber
\end{align}

\noindent \textbf{Inciso d)}
\begin{align}
    \nonumber
        h(z)        & = \frac{1}{2^{0+1}} + \frac{1}{2^{1+1}}z + \frac{1}{2^{2+1}}z^2 + \dots \\ \nonumber
                    & = \frac{1}{2^{1}} + \frac{1}{2^{2}}z + \frac{1}{2^{3}}z^2 + \dots \\ \nonumber
                    & = \frac{1}{2} \left(1 + \frac{1}{2^{1}}z + \frac{1}{2^{2}}z^2 + \dots \right)\\ \nonumber
                    & = \frac{1}{2} \left(\frac{1}{1-\frac{1}{2}z} \right)\\ \nonumber
                    & = \frac{1}{2-z} \\ \nonumber
        h'(z)       & = -\frac{\frac{d}{dz}(2-z)}{(2-z)^2} \\ \nonumber
                    & =-\frac{0-1}{(2-z)^2} \\ \nonumber
                    & =\frac{1}{(2-z)^2} \\ \nonumber
        m = h'(1)   & = \frac{1}{(2-1)^2}  \\ \nonumber
                    & = 1 \nonumber
\end{align}
Como $m \leq 1$, entonces $d=1$ y el proceso acaba con una probabilidad de $1$.

\noindent \textbf{Inciso e)}
\begin{align}
    \nonumber
        h(z)        & = \left(\frac{1}{3}\right)\left(\frac{2}{3}\right)^0 + \left(\left(\frac{1}{3}\right)\left(\frac{2}{3}\right)^1\right)z + \left(\left(\frac{1}{3}\right)\left(\frac{2}{3}\right)^2\right)z^2 + \dots \\ \nonumber
                    & = \left(\frac{1}{3}\right) + \left(\left(\frac{1}{3}\right)\left(\frac{2}{3}\right)^1\right)z + \left(\left(\frac{1}{3}\right)\left(\frac{2}{3}\right)^2\right)z^2 + \dots \\ \nonumber
                    & = \frac{1}{3} \left(1 + \left(\frac{2}{3}\right)^1z + \left(\frac{2}{3}\right)^2z^2 + \dots \right)\\ \nonumber
                    & = \frac{1}{3} \left(\frac{1}{1-\frac{2}{3}z} \right)\\ \nonumber
                    & = \frac{1}{3-2z} \\ \nonumber
        h'(z)       & = -\frac{\frac{d}{dz}(2-z)}{(2-z)^2} \\ \nonumber
                     & = -\frac{\frac{d}{dz}(3-2z)}{(3-2z)^2} \\ \nonumber
                    & =-\frac{0-2}{(3-2z)^2} \\ \nonumber
                    & =\frac{2}{(3-2z)^2} \\ \nonumber
        m = h'(1)   & = \frac{2}{(3-2)^2}  \\ \nonumber
                    & = 2 \nonumber
\end{align}
Como $m > 1$, entonces $d<1$ y el proceso acaba con una probabilidad de $d$. Para el cálculo del valor de $d$, se utliza la ecuación \ref{valord1}:
\begin{align} \label{valord1}
        z & = h{(z)}\\ \nonumber
          & = \frac{1}{3-2z} \\ \nonumber
        z({3-2z}) & = 1\\ 
        2z^2-3z+1 & = 0  \nonumber 
\end{align}
Por lo tanto, resolviendo la ecuación cuadrática queda que $z_1 = 1$ y $z_2 = 1/2 = d$.

\noindent \textbf{Inciso f)}

Considerando que $d=h{(d)=p_{0} + p_{1}d + p_{2}d^2+\dots}$, se utiliza esta expresión para estimar el valor de $d$ númericamente en el programa R, lo cual da como resultado $d \thickapprox 0.2032$. El código empleado se encuentra en el repositorio GitHub \cite{github}.

\subsection{Problema 3, página 401}
\noindent \textit{In the chain letter problem (see Example 10.14) find your expected profit if:}

\begin{enumerate}[a)]
    \item $p_{0}$ = $1/2$, $p_{1}$ = $0$, $p_{2}$ = $1/2$
    \item $p_{0}$ = $1/2$, $p_{1}$ = $0$, $p_{2}$ = $1/2$ 
\end{enumerate}
\noindent Show that if $p_{0} > 1/2$, you cannot expect to make a profit.

\noindent \textbf{Solución}

\noindent En este problema se tiene que el número esperado de cartas que se pueden vender es $m$=$p_{1}+2p_{2}$ y el valor esperado de la ganancia es $\text{E}_{(\text{ganancia})}$= $50(m+m^12)-100$, entonces se tiene que:

\noindent \textbf{Inciso a)}
    \begin{align}
    \nonumber
        m & = 0 + 2\left(\frac{1}{2}\right) \\ \nonumber
        m & = 1. \\ \nonumber
        \text{E}_{(\text{ganancia})} & = 50(1+1^{12})-100 \\ \nonumber
         \text{E}_{(\text{ganancia})} & = 0. \\ \nonumber
    \end{align}
\noindent \textbf{Inciso b)} 
    \begin{align}
    \nonumber
        m & = \frac{1}{2}+ 2\left(\frac{1}{3}\right)\\ \nonumber
        m & = \frac{7}{6}. \\ \nonumber
        \text{E}_{(\text{ganancia})} & = 50 \left(\frac{7}{6}+\left(\frac{7}{6} \right)^{12} \right)-100 \\ \nonumber
         \text{E}_{(\text{ganancia})} & \thickapprox 276.26. \\ \nonumber
    \end{align}  

\noindent \textbf{Demostración} 

\noindent Se considera que $p_{0} + p_{1} + p_{2}$ = $1$, entonces se contempla un $p_{0}$ = $0.55$, $p_{1}$ = $0.25$ y $p_{2}$ = $0.2$, y se obtiene lo siguiente:
\begin{align}
\nonumber
    m & = 0.25 + 2(0.2) \\ \nonumber
    m & = 0.65 \\ \nonumber
    \text{E}_{(\text{ganancia})} & = 50(0.65+0.65^{12})-100 \\ \nonumber
    \text{E}_{(\text{ganancia})} & \thickapprox -67.22. \\ \nonumber
\end{align}

\subsection{Problema 1, página 401}
\noindent \textit{Let $X$ be a continuous random variable with values in} [$0,2$] \textit{and density $f_{X}$. Find the moment generating function $g{(t)}$ for $X$ if:}

\begin{enumerate}[a)]
    \item $f_{X}{(x)}$ = $1/2$
    \begin{align} \label{parte1}
        \nonumber
        g{(t)}  & =  \int_{0}^{2}e^{xt}\frac{1}{2}\,dx \longrightarrow  \text{(ecuación \ref{continua})} \\ \nonumber
                & = \frac{1}{2}\int_{0}^{2}e^{xt}\,dx \\ \nonumber
                & = \frac{1}{2} \left[\left \frac{e^{xt}}{t} \right |_{0}^{2} \right] \\ \nonumber
                & = \frac{1}{2} \left[\frac{e^{2t}}{t} - \frac{e^{0t}}{t} \right] \\ 
                & = \frac{1}{2} \left[\frac{e^{2t}-1}{t} \right] \\ \nonumber
                & = \frac{e^{2t}-1}{2t}. \\ \nonumber
    \end{align}
    \item $f_{X}{(x)}$ = $(1/2)x$
    \begin{align} \label{parte2}
        \nonumber
        g{(t)}  & =  \int_{0}^{2}e^{xt}\frac{1}{2}x\,dx \longrightarrow  \text{(ecuación \ref{continua})} \\ \nonumber
                & = \frac{1}{2}\int_{0}^{2}e^{xt}x\,dx  \Longrightarrow UV - \int_{0}^{2} V \, dU, \hspace{2mm} U=x, \hspace{2mm} dU=dx, \hspace{2mm} V= \frac{e^{xt}}{t} \\ \nonumber
                & = \frac{1}{2} \left[ x \frac{e^{xt}}{t} - \int_{0}^{2}\frac{e^{xt}}{t}\,dx \right] \\ \nonumber
                & = \frac{1}{2} \left[ x \frac{e^{xt}}{t} - \frac{1}{t} \int_{0}^{2}e^{xt}\,dx \right] \\ 
                & = \frac{1}{2} \left[  x \left \frac{e^{xt}}{t} -  \frac{e^{xt}}{t^2} \right |_{0}^{2} \right] \\ \nonumber \label{parte2a}
                & = \frac{1}{2} \left[2\frac{e^{2t}}{t} -  \frac{e^{2t}}{t^2} - \left(0\frac{e^{0t}}{t} -  \frac{e^{0t}}{t^2} \right) \right] \\ \nonumber
                & = \frac{1}{2} \left[\frac{2e^{2t}}{t} -  \frac{e^{2t}}{t^2} + \frac{1}{t^2} \right] \\ \nonumber
                & = \frac{1}{2} \left[\frac{2te^{2t}-e^{2t}+1}{t^2} \right] \\ 
                & = \frac{e^{2t}(2t-1)+1}{2t^2}. \\ \nonumber
    \end{align}    
    \item $f_{X}{(x)}$ = $1-(1/2)x$
    \begin{align}
        \nonumber
        g{(t)}  & =  \int_{0}^{2}e^{xt}\left[1-\frac{1}{2}x\right]\,dx \longrightarrow \text{(ecuación \ref{continua})} \\ \nonumber
                & =  \int_{0}^{2}e^{xt}-\frac{e^{xt}}{2}x \,dx  \\ \nonumber
                & =  \int_{0}^{2}e^{xt} \,dx  - \int_{0}^{2} \frac{e^{xt}}{2}x \,dx  \\ \nonumber
                & =  \int_{0}^{2}e^{xt} \,dx  - \frac{1}{2}\int_{0}^{2} e^{xt}x \,dx  \\ \nonumber
                 & = \color{red}\frac{e^{2t}-1}{t} \color{black} - \color{blue} \frac{e^{2t}(2t-1)+1}{2t^2} \color{black} \longrightarrow (\color{red} \text{ecuación \ref{parte1}} \color{black} \text{ y } \color{blue} \text{ecuación \ref{parte2a}} \color{black} ) \\ \nonumber
                 & = \frac{2e^{2t}-2t-2e^{2t}+e^{2t}-1}{2t^2} \\ \nonumber
                 & = \frac{e^{2t}-2t-1}{2t^2}. \\ \nonumber
    \end{align}
    \item $f_{X}{(x)}$ = $|1-x|$
    \begin{align}
        \nonumber
        g{(t)}  & =  \int_{0}^{2}e^{xt}(|1-x|)\,dx \longrightarrow  \text{(ecuación \ref{continua})} \\ \nonumber
                & =  \int_{0}^{1}e^{xt}(|1-x|)\,dx + \int_{1}^{2}e^{xt}(|x-1|)\,dx \,dx  \\ \nonumber
                & =  \int_{0}^{1}e^{xt}-e^{xt}x\,dx + \int_{1}^{2}e^{xt}x-e^{xt}\,dx  \\ \nonumber 
                & = \left[\int_{0}^{1}e^{xt}-\int_{0}^{1}e^{xt}x\,dx \right] + \left[\int_{1}^{2}e^{xt}x-\int_{1}^{2}e^{xt}\,dx \right]  \\ \nonumber
                & = \left[\left[\left \color{red} \frac{e^{xt}}{t}\color{black} - \left(\color{violet} \frac{xe^{xt}}{t} - \frac{e^{xt}}{t^2} \color{black} \right) \right] \right |_{0}^{1} \right]  + \left[ \left[\left \color{violet} \frac{xe^{xt}}{t} - \frac{e^{xt}}{t^2} \color{black} - \color{red} \frac{e^{xt}}{t} \color{black} \right] \right |_{1}^{2} \right] \longrightarrow (\color{red} \text{ecuación \ref{parte1}} \color{black} \text{ y } \color{violet} \text{ecuación \ref{parte2}} \color{black})\\ \nonumber 
                & = \left[\left[\left  \frac{e^{xt}}{t} -  \frac{xe^{xt}}{t} + \frac{e^{xt}}{t^2} \right] \right |_{0}^{1} \right]  +  \left[\left[\left \frac{xe^{xt}}{t} - \frac{e^{xt}}{t^2} - \frac{e^{xt}}{t} \right] \right |_{1}^{2}\right] \\ \nonumber
                & = \left[\left[\left  \frac{e^{xt}-xe^{xt}}{t} + \frac{e^{xt}}{t^2} \right] \right |_{0}^{1} \right]  +  \left[\left[\left \frac{xe^{xt}}{t} - \frac{e^{xt}}{t^2} - \frac{e^{xt}}{t} \right] \right |_{1}^{2}\right] \\ \nonumber
                & = \left[\frac{e^{1t}-1e^{1t}}{t} + \frac{e^{1t}}{t^2} - \left(\frac{e^{0t}-0e^{0t}}{t} + \frac{e^{0t}}{t^2} \right) \right] + \left[\frac{2e^{2t}}{t} - \frac{e^{2t}}{t^2} - \frac{e^{2t}}{t} - \left(\frac{1e^{1t}}{t} - \frac{e^{1t}}{t^2} - \frac{e^{1t}}{t}  \right)\right] \\ \nonumber
                & = \left[\frac{e^{t}}{t^2} - \frac{1}{t} - \frac{1}{t^2} \right] + \left[\frac{e^{2t}}{t} - \frac{e^{2t}}{t^2} + \frac{e^{t}}{t^2} \right] \\ \nonumber
                & = \left[\frac{e^{t}-t-1}{t^2} \right] + \left[\frac{te^{2t}-e^{2t}+e^{t}}{t^2} \right] \\ \nonumber
                & = \frac{te^{2t}-e^{2t}+2e^{t}-t-1}{t^2} \\ \nonumber
                & = \frac{e^{2t}(t-1)+2e^{t}-t-1}{t^2}. \\ \nonumber
    \end{align}
    \item $f_{X}{(x)}$ = $(3/8)x^2$
    \begin{align}
        \nonumber
        g{(t)}  & =  \int_{0}^{2}e^{xt}\frac{3}{8}x^2\,dx \longrightarrow  \text{(ecuación \ref{continua})} \\ \nonumber
                & =  \int_{0}^{2}\frac{3x^2e^{xt}}{8}\,dx \\ \nonumber
                & =  \frac{3}{8} \int_{0}^{2}x^2e^{xt}\,dx \Longrightarrow UV - \int_{0}^{2} V \, dU, \hspace{2mm} U=x^2, \hspace{2mm} dU=2x\,dx, \hspace{2mm} V= \frac{e^{xt}}{t} \\ \nonumber
                & =  \frac{3}{8} \left[x^2\frac{e^{xt}}{t}-\int_{0}^{2}\frac{e^{xt}}{t}2x\,dx \right] \\ \nonumber
                & = \left[x^2\frac{e^{xt}}{t}-\frac{2}{t}\int_{0}^{2}e^{xt}x\,dx \right] \\ \nonumber
                & =  \frac{3}{8} \left[ \left \left[ x^2  \frac{e^{xt}}{t}-\frac{2}{t} \left( x \left \frac{e^{xt}}{t} -  \frac{e^{xt}}{t^2} \right)  \right] \right |_{0}^{2} \right]\\ \nonumber
                & =  \frac{3}{8} \left[ \left \left[ x^2  \frac{e^{xt}}{t}-\frac{2}{t} \left( \color{violet} x \left \frac{e^{xt}}{t} -  \frac{e^{xt}}{t^2} \color{black} \right) \color{black} \right] \right |_{0}^{2} \right] \longrightarrow (\color{violet}\text{ecuación \ref{parte2}} \color{black}) \\ \nonumber
                & =  \frac{3}{8} \left[ \left \left[\frac{x^2e^{xt}}{t}-\frac{2}{t} \left(\left \frac{txe^{xt}-e^{xt}}{t^2} \right) \right] \right |_{0}^{2} \right] \\ \nonumber
                & =  \frac{3}{8} \left[ \left \left[\frac{x^2e^{xt}}{t} - \left \frac{2txe^{xt}+2e^{xt}}{t^3} \right] \right |_{0}^{2} \right] \\ \nonumber
                & =  \frac{3}{8} \left[ \left \left[ \left \frac{t^2x^2e^{xt} - 2txe^{xt}+2e^{xt}}{t^3} \right] \right |_{0}^{2} \right] \\ \nonumber
                & =  \frac{3}{8} \left[ \frac{t^2(2)^2e^{2t} - 2t(2)e^{2t}+2e^{2t}}{t^3} - \left( \frac{t^2(0)^2e^{0t} - 2t(0)e^{0t}+2e^{0t}}{t^3} \right) \right] \\ \nonumber
                & =  \frac{3}{8} \left[ \frac{4t^2e^{2t} - 4te^{2t}+2e^{2t}-2}{t^3}  \right] \\ \nonumber
                & =  \frac{12t^2e^{2t} - 12te^{2t} + 6e^{2t} - 6}{8t^3} \\\nonumber
                & =  \frac{6t^2e^{2t} - 6te^{2t} + 3e^{2t} - 3}{4t^3}. \\ \nonumber
                & =  \frac{3}{4} \left[\frac{e^{2t}(2t^2 - 2t + 1)-1}{t^3}\right]. \\ \nonumber
    \end{align}
\end{enumerate}

\subsection{Problema 6, página 402}
\noindent \textit{Let $X$ be a continuous random variable whose characteristic function $k-{X}(\tau)$ is:}
\begin{align} \label{valork}
        k_{X}(\tau) & = e^{-|\tau|} \quad -\infty < \tau <  \infty.
\end{align}

\noindent Show directly that the density $f_{X}$ of $X$ is
\begin{align} \label{densidad}
        f_{X}{(x)} & = \frac{1}{\pi(1+x^2)}. 
\end{align}

\noindent \textbf{Solución} 

\noindent Se tiene que la función de densidad $f_{X}(x)$ es:
\begin{align}
\nonumber
    f_{X}{(x)}  & = \frac{1}{2\pi} \int_{-\infty}^{\infty}e^{-ix\tau}k_{X}(\tau)\,d\tau \\ \nonumber
                & = \frac{1}{2\pi} \int_{-\infty}^{\infty}e^{-ix\tau} \color{red} e^{-|\tau|}\, \color{black}d\tau \longrightarrow (\color{red}\text{ecuación \ref{valork}}\color{black}) \\ \nonumber
                & = \frac{1}{2\pi} \left[\int_{-\infty}^{0}e^{-ix\tau} e^{-(-\tau)}\, d\tau + \int_{0}^{\infty}e^{-ix\tau} e^{-\tau}\, d\tau \right] \\ \nonumber
                & = \frac{1}{2\pi} \left[\int_{-\infty}^{0}e^{-ix\tau+\tau} \, d\tau + \int_{0}^{\infty}e^{-ix\tau-\tau} \, d\tau \right] \\ \nonumber
                & = \frac{1}{2\pi} \left[\int_{-\infty}^{0}e^{\tau(1-ix)} \, d\tau + \int_{0}^{\infty}e^{-\tau(ix+1)} \, d\tau \right] \\ \nonumber
                & = \frac{1}{2\pi} \left[\left[\left \frac{e^{\tau(1-ix)}}{(1-ix)} \right] \right|_{-\infty}^{0}  + \left[ - \left \frac{e^{-\tau(ix+1)}}{(ix+1)} \right] \right|_{0}^{\infty} \right] \\ \nonumber
                & = \frac{1}{2\pi} \left[\left[ \frac{e^{0(1-ix)}}{(1-ix)} - \frac{e^{-\infty(1-ix)}}{(1-ix)}\right] - \left[\frac{e^{-\infty(ix+1)}}{(ix+1)} - \frac{e^{-0(ix+1)}}{(ix+1)} \right]  \right] \\ \nonumber
                & = \frac{1}{2\pi} \left[ \frac{1}{(1-ix)} + \frac{1}{(ix+1)} \right]  \\ \nonumber
                & = \frac{1}{2\pi} \left[ \frac{ix+1+1-ix}{(ix+1-(ix)^2 -ix)} \right]  \\ \nonumber
                & = \frac{1}{2\pi} \left[ \frac{2}{(1-(ix)^2)} \right]  \\ \nonumber
                & = \frac{1}{\pi} \left[ \frac{1}{(1-(\color{blue}-\color{black}x^2)} \right] \longrightarrow (\color{blue}i=\sqrt{-1}\color{black}, \, \text{(teorema 10.4)})  \\ \nonumber
                & = \frac{1}{\pi(1+x^2)} \longrightarrow \text{(ecuación \ref{densidad})}. \\ \nonumber
\end{align}

\subsection{Problema 10, página 403}
\noindent \textit{Let $X_{1}, \, X_{2}, \, \dots, \, X_{n}, \,$ be an independent trials process with density}
\begin{align} 
\nonumber
        f{(x)} & = \frac{1}{2}e^{-|x|} \quad -\infty < x <  \infty.
\end{align}

\begin{enumerate}[a)]
    \item \textit{Find the mean and variance of $f(x)$}
    \begin{align}
    \nonumber
        \mu = E{(X)} & =  \int_{-\infty}^{\infty} x \frac{1}{2}e^{-|x|} \, dx \longrightarrow (\text{ecuación \ref{Vcontinua}}) \\ \nonumber
                    & = \frac{1}{2} \left[\int_{-\infty}^{0} x e^{-(-x)} \, dx + \int_{0}^{\infty}x e^{-(x)} \, dx \right] \\ \nonumber
                    & = \frac{1}{2} \left[\int_{-\infty}^{0} x e^{(x)} \, dx + \int_{0}^{\infty}x e^{-(x)} \, dx \right] \\ \nonumber
                    & = \frac{1}{2} \left[-1 + 1 \right] \\ \nonumber
                    & = 0. \nonumber
    \end{align}
Ahora, para calcular la varianza, se necesita el valor de $E(X^2)$, entonces:
    \begin{align}
    \nonumber
        \mu = E{(X)} & =  \int_{-\infty}^{\infty} x^2 \frac{1}{2}e^{-|x|} \, dx \longrightarrow (\text{ecuación \ref{Vcontinua}}) \\ \nonumber
                    & = \frac{1}{2} \left[\int_{-\infty}^{0} x^2 e^{-(-x)} \, dx + \int_{0}^{\infty} x^2 e^{-(x)} \, dx \right] \\ \nonumber
                    & = \frac{1}{2} \left[\int_{-\infty}^{0} x^2 e^{(x)} \, dx + \int_{0}^{\infty} x^2 e^{-(x)} \, dx \right] \\ \nonumber
                    & = \frac{1}{2} \left[2+2 \right] \\ \nonumber
                    & = 2. \nonumber
    \end{align}
Por lo tanto, para calcular la varianza se utiliza la ecuación \ref{varianza}:
    \begin{align}
        \nonumber
        V{(X)}  & = 2 - (0)^2 \\ \nonumber
                & = 2 - 0 \\ \nonumber
                & = 2. \nonumber
    \end{align}
    
    \item \textit{Find the moment generating function for $X_{1}$, $S_{n}$, $A_{n}$, and $S_{n}^{*}$}
    \begin{align}
        \nonumber
        g{(t)}  & =  \int_{-\infty}^{\infty}e^{xt}\left[\frac{e^{-|x|}}{2}\right]\,dx \longrightarrow \text{(ecuación \ref{continua})} \\ \nonumber
                & = \frac{1}{2} \int_{-\infty}^{\infty}e^{xt} e^{-|x|} \,dx \\ \nonumber
                & = \frac{1}{2} \left[\int_{-\infty}^{0}e^{xt} e^{-(-x)} \,dx + \int_{0}^{\infty} e^{xt} e^{-(x)} \,dx\right] \\ \nonumber
                & = \frac{1}{2} \left[\int_{-\infty}^{0}e^{xt+x} \,dx + \int_{0}^{\infty} e^{xt-x} \,dx \right] \\ \nonumber
                & = \frac{1}{2} \left[\int_{-\infty}^{0}e^{x(t+1)} + \int_{0}^{\infty} e^{-x(-t+1)} \right] \\ \nonumber
                & = \frac{1}{2} \left[\left[\left \frac{e^{x(t+1)}}{(t+1)} \right] \right|_{-\infty}^{0}  + \left[ - \left \frac{e^{-x(1-t)}}{(1-t)} \right] \right|_{0}^{\infty} \right] \\ \nonumber
                & = \frac{1}{2} \left[\left[ \frac{e^{0(t+1)}}{(t+1)} - \frac{e^{-\infty(t+1)}}{t+1}\right] - \left[\frac{e^{-\infty(1-t)}}{(1-t)} - \frac{e^{-0(1-t)}}{(1-t)} \right]  \right] \\ \nonumber
                & = \frac{1}{2} \left[ \frac{1}{(t+1)} + \frac{1}{(1-t)} \right]  \\ \nonumber
                & = \frac{1}{2} \left[ \frac{1-t+t+1}{(t-t^2+1-t)} \right]  \\ \nonumber
                & = \frac{1}{2} \left[ \frac{2}{(1-t^2)} \right]  \\ \nonumber
                & = \frac{1}{(1-t^2)}. \\ \nonumber
    \end{align}
    \begin{align}
        \nonumber
        S_{n}  & = (g{(t)})^n \\ \nonumber
                & = \left(\frac{1}{(1-t^2)} \right)^n \\ \nonumber
                & = \frac{1}{(1-t^2)^n}. \\ \nonumber
    \end{align}
        \begin{align}
        \nonumber
        S_{n}^*  & = \left(g{\left(\frac{t}{\sqrt{n}}\right)}\right)^n \\ \nonumber
                & = \left(\frac{1}{\left(1-\left(\frac{t}{\sqrt{n}}\right)^2\right)} \right)^n \\ \nonumber
                & = \frac{1}{\left(1-\left(\frac{t}{\sqrt{n}}\right)^2\right)^n}.  \\ \nonumber
    \end{align}
    \item \textit{What can you say about the moment generating function of $S_{n}^{*}$ as $n \longrightarrow \infty$?}
        \begin{align}
        \nonumber
        \lim_{n \to \infty}\frac{1}{\left(1-\left(\frac{t}{\sqrt{n}}\right)^2\right)^n}\nonumber
    \end{align}
Por lo tanto, la función generadora de  $S_{n}^{*} \longrightarrow 1$.
\end{enumerate}



\bibliography{refProbabilidad}
\bibliographystyle{plain}

\end{document}
