\documentclass{article}

\usepackage{geometry}
\geometry{a4paper}
\setlength{\parindent}{10mm}
\setlength{\parskip}{0.9em}
\def\baselinestretch{1.5}

\usepackage[spanish,mexico]{babel}
\renewcommand {\spanishtablename}{Cuadro}
\usepackage[spanish,onelanguage,ruled]{algorithm2e}
\usepackage[utf8]{inputenc}
\usepackage{graphicx}
\usepackage{caption} 
\usepackage{amsmath}
\usepackage{amsthm, amsfonts}
\usepackage{enumerate} 
\usepackage{fancyhdr}
\usepackage{anysize} 
\usepackage[usenames]{color}
\usepackage{booktabs}
\usepackage{etoolbox}
\usepackage{fancyvrb}
\usepackage{color,soul}
\usepackage[dvipsnames]{xcolor}
\usepackage{graphicx}
 \usepackage{caption}
\usepackage{subcaption}
\usepackage{listings}
\usepackage{amssymb}
\usepackage{multirow}
\usepackage{dirtytalk}



\usepackage{verbatim}
% redefine \VerbatimInput
\RecustomVerbatimCommand{\VerbatimInput}{VerbatimInput}%
 {fontsize=\footnotesize,
  %
  frame=lines,  % top and bottom rule only
  framesep=1em, % separation between frame and text
  rulecolor=\color{Gray},
  %
  label=\fbox{\color{Black}test.txt},
  labelposition=topline,
  %
  %commandchars=\ \mid \(\), % escape character and argument delimiters for
                  % commands within the verbatim
  %commentchar=*        % comment character
 }


\pagestyle{fancy}

\chead{}
\lhead{} 
\rhead{}
\lfoot{\it }
\cfoot{}
\rfoot{\thepage}

\title{
\centering
Modelos Probabilistas Aplicados \\
Johanna Bolaños Zúñiga \\
Matricula: 1883900\\
Tarea 16
}

\date{}

\begin{document}
\maketitle

\section{Retroalimentación para propuestas del proyecto final de compañeros }

\begin{enumerate}
    \item \textbf{Propuesta $1$ de Alberto Moa:} Es un interesante tema el cual te ayuda a enriquecer los resultados obtenidos y creo que estaría bien que trataras de implementar un diseño Taguchi y contrastar tus resultados.
    \item \textbf{Propuesta $2$ de Erick:} Interesante tema el cual ceo que podrías complementar utilizando grafos aleatorios y llevar a cabo una experimentación para analizar el efecto de agregar los nodos de transbordo.
    \item \textbf{Propuesta $1$ de Gabriela:} También podrías usar las pruebas de hipótesis de medias de diferencia para determinar que tanto mejoran las soluciones encontradas con los diferentes tipos de combinaciones.
\end{enumerate}


%\bibliography{refProbabilidad}
%\bibliographystyle{plain}

\end{document}

